%%%%%%%%%%%%%%%%%%%%%%%%%%%%%%%%%%%%%%%%%%%%
% https://github.com/martinhelso/uioposter %
%%%%%%%%%%%%%%%%%%%%%%%%%%%%%%%%%%%%%%%%%%%%
% Class options                            %
%%%%%%%%%%%%%%%%%%%%%%%%%%%%%%%%%%%%%%%%%%%%
% Orientation:                             %
% portrait (default), landscape            %
%                                          %
% Paper size:                              %
% a0paper (default), a1paper, a2paper,     %
% a3paper, a4paper, a5paper, a6paper       %
%                                          %
% Language:                                %
% english (default), norsk                 %
%%%%%%%%%%%%%%%%%%%%%%%%%%%%%%%%%%%%%%%%%%%%
\documentclass[a0paper]{uioposter}


\usepackage{lipsum}                                % Dummy text
\usepackage[figwidth = 0.98\linewidth]{todonotes}  % Dummy image (and more!)
\usepackage[absolute, overlay]{textpos}            % Figure placement
\usepackage{polyglossia}
\setmainlanguage{czech}
\setotherlanguages{english}
\usepackage[mono=true]{libertinus-otf}
\usepackage{microtype}
\setlength{\TPHorizModule}{\paperwidth}
\setlength{\TPVertModule}{\paperheight}


\title{Konceptuální modelování pomocí schematických kategorií}
\author {Dennis Pražák}
%% Optional:
%\institute
%{
%    \inst{1} Department of Mathematics
%    \and
%    \inst{2} Department of Informatics
%}
% Or:
%\institute{Contact information}


%% Remove footline:
%\setbeamertemplate{footline}{}


\begin{document}
\begin{frame}
\begin{columns}[onlytextwidth]


\begin{column}{0.5\textwidth - 1.5cm}
    \begin{block}{Úvod}
        Konceptuální modely jako ER a UML byly vyvinuty v době, kdy měl největší zastoupení v databázových systémech jeden logický model -- relační.
        V dnešní době se však kromě relačního modelu ve vyšší míře používá mnoho odlišných logických modelů (grafový, dokumentový, key-value, wide column\dots).

        \alert{Schematické kategorie} jsou nový prostředek ke konceptuálnímu modelování, který je obecnější (má vyšší vyjadřovací schopnost) a není natolik provázaný s logickou vrstvou.
        Schematické kategorie splňují vlastnosti teorie kategorií a lze na ně tak aplikovat již vybudované nástroje tohoto oboru matematiky.

        Cílem této práce bylo vytvořit webovou aplikaci, která umožní konceptuální modelování pomocí schematických kategorií v již známém modelu Entity-Relationship.
        ER schéma se automaticky převede na schematické kategorie.
        Usnadní tak výzkumníkům a potenciálním uživatelům schematických kategorií prozkoumání a seznámení se s tímto novým konceptem.
    \end{block}

    \begin{alertblock}{How do you make it pop?}
        Use an \alert{alertblock}!
    \end{alertblock}

    \begin{block}{Method}
        \lipsum[1]
    \end{block}

    \begin{block}{Results}
        \lipsum[2]
        \missingfigure{Striking imagery relevant to the research}
        \unskip
    \end{block}
\end{column}


\begin{column}{0.5\textwidth - 1.5cm}
    \begin{block}{Conclusions}
        \lipsum[4]
    \end{block}

    \begin{block}{Acknowledgements}
        \lipsum[5]
    \end{block}

    \begin{block}{References}
        \lipsum[6]
    \end{block}

    \begin{block}{Kontakt}
        \lipsum[75]
    \end{block}
\end{column}


\end{columns}


\begin{textblock}{0.5}(0.50, 0.94)
    \color{white}
    \sffamily
    \textbf{Write here using textblock}
    \\
    Such as contact information or references
\end{textblock}


\end{frame}
\end{document}
