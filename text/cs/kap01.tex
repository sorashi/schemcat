\chapter{Existující nástroje}

V~této kapitole zanalyzujeme některé existující nástroje pro tvorbu diagramů
a~porovnáme je dle navržených kritérií. Nástroji, které budeme porovnávat jsou
\begin{itemize}
  \item diagrams.net~\cite{diagramsnet21} vhodné pro tvorbu libovolných diagramů,
  \item drawSQL~\cite{drawsql21} určené pro tvorbu relačních schémat,
  \item ERDPlus~\cite{erdplus21} k~vytváření zejména ER diagramů~\cite{Chen76}.
\end{itemize}

Před představením existujících nástrojů určíme srovnávací kritéria, dle kterých
budeme nástroje analyzovat.

\section{Srovnávací kritéria}

Prvním kritériem pro porovnání nástrojů je jejich kategorie, která vypovídá
o~účelu nástroje a~cílové skupině zákazníků. Základní kategorie jsou
\begin{itemize}
  \item konceptuální vrstva -- tyto nástroje jsou většinou určené pro tvorbu ER
  diagramů, případně jiným způsobem modelují vztahy a~atributy entit, na které
  při datovém modelování vymezujeme svůj diskurz,
  \item logická (též technologická) vrstva -- tyto nástroje umožňují tvorbu
  diagramů s~ohledem na typ struktur, v~kterých jsou data uchovávána, např.
  relační databáze,
  \item kresba libovolných diagramů -- nástroje, které nejsou omezeny téměř
  žádným standardem či konvencí a~umožňují kresbu libovolných diagramů,
  \item kresba omezených diagramů -- nástroje, které umožňují kresbu diagramů
  omezených na existující schémata (ER, UML~\cite{uml2017}, \dots).
\end{itemize}

Dalším kritériem je typ úložiště. Nástroje mohou ukládat svá data do paměti
prohlížeče (lokálně pro uživatele), na své servery, nebo používat externí
úložiště uživatele, například Google
Drive\footnote{\url{https://www.google.com/drive/}}. Čím více různých typů
úložiště nástroj podporuje, tím lépe, neboť uživatel může flexibilně zvolit jeho
účelům vyhovující způsob uchovávání dat. Pro interaktivní spolupráci s~týmem je
lepší sdílené úložiště a~pro lokální práci je vhodnější lokální úložiště.

Interaktivní spolupráce je dalším důležitým kritériem. U~velkých projektů je
vývoj modelu urychlen, pokud nástroj spolupráci umožňuje.

Dále budeme porovnávat formát, do kterého nástroj diagram ukládá (pokud
k~uloženému souboru má uživatel přístup). Může se jednat o~serializovaný dokument
do dobře známého standardního formátu, nebo o~vlastní formát, který je často
nakonec také založený na nějakém standardu.

Kromě uložení rozdělané práce do vhodného formátu musí nástroj umožnit export do
formátu, který uživatelé využijí pro své účely. Formáty pro export lze rozdělit
do několika kategorií:
\begin{itemize}
  \item serializovaný formát -- většinou se jedná o~vlastní formát aplikace
a~takový soubor nelze jinou aplikací otevřít, ale lze jej programově zpracovat,
  \item rastrové formáty, např. PNG\footnote{Portable Network Graphics --
  \url{https://www.w3.org/TR/2003/REC-PNG-20031110/}} -- mají nejširší využití
  a~podporu, lze je použít v~dokumentech a~na webových stránkách,
  \item vektorové formáty, např. SVG~\cite{Dirk18} -- nemají tak rozšířenou
  podporu, nicméně jsou vhodnější v~dokumentech po estetické stránce (zvlášť při
  tištění); dále existují vektorové editory, pomocí nichž lze výsledek libovolně
  upravovat bez potřeby souboru ve serializovaném formátu; většina webových
  prohlížečů formát SVG podporuje a~soubor vykreslí; do této kategorie lze
  zařadit i~jiné otevřené strukturované formáty, např. VSDX\footnote{Microsoft
  Visio XML formát založený na ISO 29500 --
  \url{https://interoperability.blob.core.windows.net/files/MS-VSDX/\%5bMS-VSDX\%5d.pdf}},
  \item zjednodušený export -- některé nástroje šetří práci uživatele tím, že
  diagram rovnou exportují do HTML\footnote{HyperText Markup Language --
  \url{https://w3.org/TR/2021/SPSD-html52-20210128/}}, PDF\footnote{Portable
  Document Format, ISO 32000 -- \url{https://iso.org/standard/75839.html}}
  a~podobných finálních formátů pro okamžitou aplikaci, přestože uživatel může
  zvolit jiný formát a~finální vytvořit sám,
  \item schématické formáty, např. SQL\footnote{Structured Query Language --
  \url{https://iso.org/standard/63555.html}} -- téměř výhradně u~nástrojů
  lo\-gic\-ké vrst\-vy; umožňují rovnou vytvářet schémata pro databáze.
\end{itemize}

Stejně jako u~typu úložiště, čím více různých formátů exportu nástroj podporuje,
tím lépe, neboť nástroj je flexibilní.

Posledním, neméně důležitým kritériem, je způsob komercializace. Většina volně
dostupných nástrojů je nějakým způsobem zpoplatněna, ať už se jedná
o~jednorázový nebo pravidelný poplatek. Nejčastějším komerčním modelem je verze
zdarma s~omezenými funkcemi a~dále několik placených plánů různé úrovně
s~odemčenými pokročilými funkcemi. U~tohoto modelu je důležité vyrovnat funkce
tak, aby byl nástroj použitelný i~v~bezplatné verzi, a~aby byly placené funkce
atraktivní pro uživatele. Při srovnávání budeme věnovat pozornost i~tomu, jestli
jsou placené funkce esenciální.

\section{diagrams.net}
\label{section:digramsnet}

Srovnávací kritéria:
\begin{itemize}
  \item kategorie -- kresba libovolných diagramů,
  \item typ úložiště -- lokální, externí, prohlížeč,
  \item export -- serializovaný, rastrový, vektorový, zjednodušený,
  \item interaktivní spolupráce -- částečně podporována (pomocí externích úložišť),
  \item komercializace -- veškeré funkce jsou zdarma a~není potřeba uživatelský
  účet; z~jiného pohledu lze počítat cenu externích úložišť, ale ta jsou
  volitelná.
\end{itemize}

Nástroj diagrams.net~\cite{diagramsnet21}, dříve draw.io, je obecný open-source
kreslící nástroj (který však nepřijímá změny od externích vývojářů) vydaný
s~licencí Apache License
2.0\footnote{\url{https://www.apache.org/licenses/LICENSE-2.0}}, dostupný jako
webová aplikace\footnote{na adrese \url{https://app.diagrams.net}} nebo jako
desktopová aplikace. Desktopová verze aplikace je sestavena stejným způsobem
jako webová, pouze je zabalena pomocí platformy Electron~\cite{electron21}  do
okna Chromium. Je vyvinut v~běžných we\-bo\-vých tech\-no\-lo\-gi\-ích
(Java\-Script\footnote{Standardizován jako ECMAScript, ISO 16262 --
\url{https://iso.org/standard/55755.html}}, CSS\footnote{Cascading Style Sheets --
\url{https://www.w3.org/TR/css}}, HTML).

Diagramy lze uložit do serializovaného XML\footnote{Extensible Markup Language
-- \url{https://www.w3.org/TR/xml/}} formátu .drawio. V~tomto formátu je pro
každý diagram XML element \texttt{diagram}, ve kterém se nachází data zakódována
do Base64\footnote{RFC 2045 \S6.8 --
\url{https://datatracker.ietf.org/doc/html/rfc2045\#section-6.8}}. Tato data
jsou komprimována pomocí zlib\footnote{\url{https://zlib.net}} a~obsahují další
XML dokument (URL-encoded\footnote{RFC 3986 \S2.1 --
\url{https://datatracker.ietf.org/doc/html/rfc3986\#section-2.1}}, tj.
zakódovaný), tentokrát již serializaci vlastního diagramu. Formát tak není bez
dekomprese čitelný člověkem. Výhodou je, že lze uložit více diagramů do jednoho
souboru a~každý pojmenovat. Rozhraní k~tomu určené je identické s~listy souboru
tabulkových procesorů, jako Microsoft Excel\footnote{\url{https://aka.ms/excel}}
a~Google Sheets\footnote{\url{https://sheets.google.com}}.

Soubor s~diagramy lze také uložit do formátu SVG, který je navíc otevřený
a~podporují ho jiné nástroje. Uživatel má při exportu k~dispozici možnost
\textit{Include a~copy of my diagram}, která do SVG souboru zahrne již zmíněný
Base64 řetězec, ve kterém je diagram serializovaný. Ve výsledku to znamená, že
takto exportované SVG soubory umí diagrams.net i~otevřít a~práce na nich může
plnohodnotně pokračovat. Toto řešení se nám líbí, protože se jedná o~schování
vlastního formátu do SVG, který je nejvhodnějším pro přechovávání a~zobrazování
diagramů.

Dalšími možnostmi exportu a~ukládání jsou
\begin{itemize}
  \item rastrové soubory PNG, JPEG\footnote{Joint Photographic Experts Group,
  ISO 19566 -- \url{https://iso.org/standard/65348.html}},
  \item soubor PDF, do kterého je ve vektorovém formátu diagram vložen,
  \item soubor HTML, do kterého lze podobně jako v~SVG data diagramu uložit
v~serializované formě, případně pouze vložit veřejný odkaz URL na diagram (pokud
je použito odpovídající úložiště); v~tomto souboru je pak zahrnut JavaScript od
diagrams.net, který diagram vykreslí,
  \item otevřený formát VSDX, původně vyvinutý pro Microsoft Visio.
\end{itemize}

Ze stejných souborů lze diagramy také importovat, ovšem editovat je lze jen
pokud je v~nich zahrnut formát drawio, čehož je dosaženo u~některých formátů
popsaných výše.

Jako úložiště si lze vybrat Google Drive,
OneDrive\footnote{\url{https://aka.ms/onedrive}},
Dropbox\footnote{\url{https://dropbox.com}},
GitHub\footnote{\url{https://github.com}},
GitLab\footnote{\url{https://gitlab.com}}, paměť prohlížeče a~místní úložiště
(disk uživatele). Soubor lze ze stejných úložišť i~otevřít a~importovat, navíc
k~tomu i~z~libovolné dostupné URL.

Interaktivní spolupráce je umožněna pouze pokud soubor jako úložiště využívá
takové, ke kterému mají přístup zápisu (popř. pouze čtení) všichni účastnící se
uživatelé (Google Drive, OneDrive, Dropbox, GitHub, GitLab). Tato úložiště je
však nutno manuálně vhodně nastavit (přístup ostatním uživatelům). U~všech
úložišť je rychlost reflektování změn ostatních uživatelů podobná -- vcelku
pomalá, protože aplikace musí změny aktivně kontrolovat a~načítat.

Menu \texttt{File $\rightarrow$ Publish} chybně napovídá, že se jedná o~funkci
interaktivní spolupráce. Ve skutečnosti je uživateli jen zobrazen odkaz na
soubor ve vybraném úložišti (ale pouze pro Google Drive a~OneDrive, jinak je
tato možnost vypnuta). Spolupracující uživatel tak musí tento soubor v~daném
úložišti uložit k~sobě (sdíleně), aby mohla spolupráce začít.

Jako další možnost jsme zvažovali desktopovou aplikaci s~načteným souborem,
který je libovolným externím nástrojem sdílen mezi uživateli. Bohužel, soubor se
nepřenačítá automaticky, ale musí být manuálně synchronizován tlačítkem
\texttt{File $\rightarrow$ Synchronize (Alt+Shift+S)}, které je dostupné pouze
v~desktopové verzi aplikace. Uživatel je při externí změně souboru upozorněn
(avšak ne spolehlivě vždy) červeným nápisem. Algoritmus synchronizace funguje
správně a~tak, jak uživatel očekává.

Nejlepší způsob dosažení interaktivní spolupráce je dle našeho názoru volba
systému pro správu Git\footnote{Systém pro správu verzí Git --
\url{https://git-scm.com}} repozitářů (GitLab nebo GitHub), protože
\begin{enumerate}
  \item tato úložiště jsou dostupná jak z~webové, tak z~desktopové verze
  aplikace,
  \item synchronizace probíhá pomocí systému Git,
  \item díky použití systému Git lze jednoduše spravovat verze a~body v~historii
  při vývoji diagramu.
\end{enumerate} 

K~poslednímu bodu je třeba podotknout, že jiná webová úložiště také podporují
správu verzí, avšak není tak rozvinutá, jako správa systémem k~tomu určeným --
Git. Diagrams.net sám o~sobě správu verzí neobsahuje, jen obvyklé ``Undo, Redo''
pro aktuálního uživatele. Úpravy ostatních uživatelů nelze vracet postupně, lze
se pouze vrátit za bod synchronizace.

Uživateli jsou v~levém postranním panelu k~dispozici standardní tvary ER
diagramů, UML diagramů~\cite{uml2017}, flowchart diagramů a~další základní tvary
pro kresbu diagramů. Tvary lze libovolně kombinovat a~spojovat podržením levého
tlačítka a~tažením myší z~a~do kotev na krajích objektů. Každý objekt
a~spojovací čára má vlastnosti, které lze upravovat v~pravém postranním panelu.
Upravovat lze přímo i~vlastnosti formátu SVG.

Uživatelské rozhraní, které je vidět na obrázku \ref{fig:diagrams.net}, je velmi
podobné kancelářským aplikacím Google. Je tak přívětivé pro nové uživatele,
kteří již s~aplikacemi Google dříve pracovali.

Jako výhody určujeme
\begin{itemize}
  \item univerzálnost a~flexibilita -- nástroj lze použít pro tvorbu jakýchkoli
  diagramů,
  \item množství podporovaných formátů -- export pokrývá téměř všechny možné
  účely,
  \item cena -- všechny funkce jsou zdarma,
  \item více diagramů v~jednom souboru
\end{itemize}
a~nevýhodami jsou
\begin{itemize}
  \item chybějící možnost pro export do (jednoduše) strojově zpracovatelného
  formátu, nelze tak bez lidské práce diagram převést do logické vrstvy (to je
  zapříčiněno obecností nástroje, jeho účelem je kresba, ne abstrakce),
  \item pomalé zobrazování změn při interaktivní spolupráci, zároveň není
  zpočátku jasné, jak spolupráce dosáhnout.
\end{itemize}

Výhodou i~nevýhodou může být nutnost použití externího úložiště. Pro velké
společnosti se může jednat o~bezpečnostní opatření, protože diagrams.net
k~diagramům nemá přístup. Pro malé týmy se může jednat o~nevýhodu, protože je
potřeba účet na externím webu, nebo jiný způsob sdílení a~správa tohoto
úložiště.

\begin{figure}
  \centering
  \includegraphics[width=\textwidth]{../img/diagrams.net.png}
  \caption{Tvorba ER diagramu v~aplikaci diagrams.net}
  \label{fig:diagrams.net}
\end{figure}

\section{drawSQL}

Srovnávací kritéria:
\begin{itemize}
  \item kategorie -- logická vrstva,
  \item typ úložiště -- online, poskytované autory produktu
  \item export -- schématický (obecný SQL i~platformě specifické formáty),
  rastrový PNG, serializovaný (JSON~\cite{json2017}, v~době psaní práce se
  chystá)
  \item interaktivní spolupráce -- pouze v~placené verzi,
  \item komercializace -- omezená verze navždy zdarma, různé měsíčně placené
  plány.
\end{itemize}

Nástroj drawSQL~\cite{drawsql21} je modelovací nástroj pro tvorbu relačních
schémat. Aplikace je dostupná ve webovém prohlížeči\footnote{na adrese
\url{https://drawsql.app}}. Je vyvinuta ve standardních webových technologiích
a~používá framework Vue.js. Plán zdarma umožňuje tvorbu veřejně přístupných
diagramů, které mohou mít maximálně 15 tabulek (entit). Měsíčně placené plány
umožňují vytvářet neveřejné diagramy, více (až neomezeně mnoho) tabulek
v~diagramu, více uživatelů, kteří mohou na diagramu spolupracovat, a~přístup
k~verzovacím nástrojům. K~vyzkoušení i~používání nástroje je potřeba uživatelský
účet.

Hlavní funkcí drawSQL je export schématu do SQL. Proto si uživatel při vytváření
diagramu zvolí cílovou databázi, pro kterou schéma tvoří. Výsledné SQL tak bude
mít tvar, se kterou cílová databáze umí pracovat. Podporovanými databázemi jsou
MySQL\footnote{\url{https://mysql.com}},
PostgreSQL\footnote{\url{https://postgresql.org}} a~SQL
Server\footnote{Microsoft SQL Server -- \url{https://aka.ms/sqlserver}}.

Rozhraní, které je vidět na obrázku \ref{fig:drawsql}, obsahuje diagram
a~postranní panel. V~postranním panelu lze vytvářet jednotlivé tabulky,
definovat jejich sloupce a~vlastnosti jednotlivých sloupců -- typ sloupce,
nullability\footnote{\emph{nullability} je příznak, který určuje, zda lze sloupec
v~řádku nastavit na hodnotu \texttt{NULL}}, zda se jedná o~primární klíč, unikátní klíč
nebo index. Tyto změny se v~reálném čase reflektují v~diagramu, ve kterém může
uživatel jednotlivé sloupce spojovat, čímž vytváří cizí klíče. Pozici těchto
lomených čar lze upravovat pouze posunutím tabulky v~diagramu. Pokud je cizích
klíčů víc, začne být diagram velmi nepřehledný.

Diagram lze importovat ze souboru SQL stisknutím \texttt{File $\rightarrow$
Import}. Stisknutím tlačítka \texttt{File $\rightarrow$ Export} se otevře
nabídka Export, ve které může uživatel diagram exportovat do SQL své předem
zvolené databáze, nebo do rastrového obrázku ve formátu PNG. Vývojáři aplikace
plánují implementovat také export diagramu pomocí serializace do formátu JSON.
V~nabídce Export je navíc možnost nechat si vygenerovat platformně specifický
kód jako například migrační třídy pro Laravel\footnote{Framework pro PHP --
\url{https://laravel.com}}, definice modelů pro Laravel a~migrační schémata pro
AdonisJS\footnote{Framework pro Node.js -- \url{https://adonisjs.com}}.

Interaktivní spolupráce je k~dispozici pouze v~placené verzi. Dle našeho názoru
je interaktivní spolupráce hlavní funkcí tohoto nástroje oproti konkurenčním
relačním modelovacím nástrojům. Některá integrovaná vývojová prostředí (např.
Visual Studio\footnote{Vývojové prostředí Microsoft Visual Studio --
\url{https://visualstudio.microsoft.com}}) obsahují nástroj pro relační
modelování i~generování databázového schématu. Hlavním omezením těchto nástrojů
je však absence interaktivní spolupráce, jedná se spíše o~spolupráci iterací.
Proto považujeme určení interaktivní spolupráce za placenou funkci za negativní
rozhodnutí pro využitelnost nástroje v~relaci s~konkurencí.

Web drawSQL také zveřejňuje šablony modelů\footnote{na adrese
\url{https://drawsql.app/templates}} (jedná se spíše o~příklady). Šablony jsou
většinou potenciální modely známých produktů (např.
WordPress\footnote{\url{https://wordpress.com}}) a~tvoří je autoři drawSQL.
Tuto funkci považujeme za výhodu, protože společnosti a~individuální vývojáři se
mohou inspirovat existujícími a~ověřenými řešeními, případně nezačínat se svým
modelem od nuly.

Závěrem určíme výhody drawSQL:
\begin{itemize}
  \item příjemné uživatelské rozhraní (viz obrázek \ref{fig:drawsql}),
  \item možnost určení typu relace, o~sémantiku se aplikace stará sama
  (one-to-one, one-to-many, many-to-many),
  \item několik platformě specifických generátorů modelu,
  \item šablony a~příklady existujících modelů
\end{itemize}
a~nevýhody:
\begin{itemize}
  \item nelze upravit ani přesunout lomené čáry spojující cizí klíče, což
  způsobuje chaos pokud je v~diagramu větší množství entit,
  \item interaktivní spolupráce pouze v~placeném plánu,
  \item správa verzí pouze v~placeném plánu,
  \item k~vyzkoušení nástroje je potřeba uživatelský účet,
  \item podporuje pouze relační databáze.
\end{itemize}

\begin{figure}
  \centering
  \includegraphics[width=\textwidth]{../img/drawsql.png}
  \caption{Tvorba diagramu v~drawSQL}
  \label{fig:drawsql}
\end{figure}

\section{ERDPlus}
\begin{itemize}
  \item kategorie -- logická vrstva,
  \item typ úložiště -- online, poskytované autory produktu
  \item export -- rastrový PNG,
  \item interaktivní spolupráce -- není,
  \item komercializace -- zdarma.
\end{itemize}

Nástroj ERDPlus~\cite{erdplus21} je modelovací nástroj pro tvorbu ER diagramů,
relačních schémat a~hvězdicových schémat. Aplikace je dostupná ve webovém
prohlížeči\footnote{na adrese \url{https://erdplus.com}}. Její uživatelské
rozhraní je tedy vyvinuto ve standardních webových technologiích -- HTML, CSS
a~JavaScript -- a~dále využívá framework React~\cite{react2021} pro tvorbu
rozhraní v~jazyce JavaScript.

ERDPlus lze používat bez založení uživatelského účtu a~vytvořený diagram
exportovat do speciálního formátu erdplus, nicméně uživatel tak přijde o~možnost
využití úložiště diagramů na serveru aplikace. Diagramy (ERDPlus je nazývá
\emph{dokumenty}) lze organizovat do složek a podsložek. Služby ERDPlus včetně
úložiště nejsou žádným způsobem zpoplatněny.

Tvorba ER diagramů je intuitivní s~jednoduchým uživatelským rozhraním, které je
vidět na obrázku \ref{fig:erdplus}. Uživatel má na výběr mezi vytvořením entity,
atributu, relace, spojení mezi těmito objekty a~jednoduchého textového popisku.
V~pravé části rozhraní se nachází panel s~vlastnostmi zvoleného objektu. V~tomto
panelu může uživatel také rychleji tvořit atributy entit a~relací. Při zvolení
relace lze v~panelu zvolit entity, které mají být v~relaci, a~spojení je pak
automaticky vytvořeno. Zároveň lze zvolit jednotlivé multiplicity relace.

Soubor s~diagramem je v~úložišti reprezentován vlastním formátem erdplus. Jedná
se o~textový soubor, jehož obsahem je JSON reprezentace diagramu. Diagram lze
exportovat do rastrového formátu PNG.

Zajímavou funkcí je také převod do relačního schématu. Tato funkce je dostupná
pouze tehdy, když uživatel ER diagram uloží na server ERDPlus. Poté zvolí
možnost \emph{Convert to Relational Schema} a~ERDPlus vytvoří nové relační
schéma. Z~relačních schémat lze podobně vygenerovat SQL.

Vlastnoruční tvorba relačních diagramů probíhá podobně. Uživatel může tvořit
tabulky, přidávat jim sloupce a v~tabulkách volit primární klíče v~postranním
panelu. Pomocí tlačítka \texttt{Connect} lze poté přidat cizí klíč, který
odkazuje do jiné tabulky tažením myši. ERDPlus do tabulky přidá všechny primární
klíče, které cílová tabulka obsahuje, jako nové sloupce. K~vytvoření cizího
klíče, který odkazuje na stejnou tabulku (tzv. rekurzivní klíč) slouží tlačítko
\texttt{Recursive Key} ve vlastnostech tabulky. Hvězdicovým schématům se věnovat
nebudeme, protože jsou mimo rozsah této práce.

Výhody:
\begin{itemize}
  \item převod diagramu z~ER do relačního diagramu,
  \item jednoduchost a intuitivnost procesu kresby diagramu 
\end{itemize}
a nevýhody:
\begin{itemize}
  \item relační diagram bývá nepřehledný, nelze měnit pořadí jednotlivých
  definovaných sloupců v~tabulce,
  \item chybí vektorový export,
  \item diagramy nelze stylizovat.
\end{itemize}

\begin{figure}
  \centering
  \includegraphics[width=\textwidth]{../img/erdplus.png}
  \caption{Tvorba ER diagramu v~ERDplus}
  \label{fig:erdplus}
\end{figure}

\section{nomnoml}

\begin{itemize}
  \item kategorie -- kresba omezených diagramů -- UML,
  \item typ úložiště -- online,,
  \item export -- rastrový PNG, vektorový SVG,
  \item interaktivní spolupráce -- není k~dispozici,
  \item komercializace -- zdarma s~otevřeným zdrojovým kódem.
\end{itemize}

Nástroj nomnoml je modelovací nástroj pro tvorbu UML diagramů dostupný ve
webovém prohlížeči\footnote{na adrese \url{https://nomnoml.com}}. Jeho klíčová
vlastnost je, že místo interakce myší s~webovou aplikací probíhá kresba
deklarativně -- psaním.

Uživatelské rozhraní (viz obrázek \ref{fig:nomnoml}) se skládá z~oblasti pro
textový vstup, nad kterou je zároveň (v~reálném čase) vykreslován výsledný
diagram. V~pravé horní části se nachází několik tlačítek, při kliknutí na
některé z~nich se vždy otevře pravý postranní panel s~odpovídajícími informacemi
a funkcemi. První tlačítko ukazuje rychlý přehled jazyka, ve kterém se má
diagram definovat. Druhé tlačítko odhalí kompletní referenci k~tomuto jazyku.
Dále lze najít tlačítka pro export, sdílení a uložení do místního úložiště
uživatele.

Jazyk diagramů je velmi jednoduchý. Skládá se z~definic entit a jejich relací.
Uživatel může vyjít z~úvodního diagramu, který se ukáže při navštívení hlavní
stránky nástroje. Pro ukázku, definice entity vypadá následovně

\noindent\texttt{[<abstract> Entita|soukromaSlozka; soukromaSlozka2|verejnyAtribut]}.

Svislá čára odděluje kategorie atributů, může jich být neomezené množství.
Entita je definována jako abstraktní, což také ovlivní její výsledný styl
vzhledu v~diagramu.

Dále se v~jazyce definují vztahy mezi entitami následovně
\texttt{[Entita]->[Entita2]}. Různé šipky mají různé významy. Vztah \texttt{->}
je \emph{asociace}, dále \texttt{o->} je \emph{agregace}, apod.

V~jazyce lze také deklarovat direktivy, začínající znakem \texttt{\#}. Těmi lze
upravit vzhled, vytvořit nové styly, a nastavit algoritmy, kterými bude zvoleno
rozložení entit v~diagramu. Algoritmy lze nastavit direktivou \texttt{\#ranker}
a na výběr je ze tří možností: \texttt{network-simplex}, \texttt{tight-tree},
\texttt{longest-path}. Nástroj nomnoml používá k~vykreslování diagramu knihovnu
dagre\footnote{\url{https://github.com/dagrejs/dagre}} pro JavaScript, jejíž
vývoj byl však ukončen. Z~dokumentace této knihovny vyplývá, že možnost
\emph{ranker} mění algoritmus, který vrcholům v~grafu (diagramu) přiřazuje
důležitost, která se pak odráží v~pořadí zobrazení entit. Definitivní význam
této možnosti není z~dokumentace zřejmý, u~nástroje nomnoml lze však pozorovat
změnu v~pořadí a vzdálenostech entit (délce spojovacích čar). Uživatel tak může
vyzkoušet různá rozložení a použít to, které je vizuálně nejpřehlednější.

Diagram lze exportovat do formátu PNG a dále podobně jako v~sekci
\ref{section:digramsnet}, nomnoml také umožňuje export do SVG se zakomponovaným
zdrojovým kódem. Uživatel tak může diagram distribuovat v~tomto formátu a
zároveň tento formát v~nástroji nomnoml i otevřít a plnohodnotně pokračovat
v~práci. Podobně je možné sdílet odkaz přímo na vytvořený diagram ve službě
nomnoml. Ten je vytvořen tak, že do odkazu URL je jako parametr \texttt{source}
zapsán přímo zdrojový kód diagramu. Výsledná adresa URL je tak velice dlouhá,
ale služba nomnoml nemusí diagramy ukládat, stačí jej vykreslit z~dat v~odkazu.
Přestože se nejedná o~opravdové online úložiště, kategorizovali jsme ho tímto
způsobem. Rozdělaný diagram lze také uložit do paměti prohlížeče.

Nástroj nomnoml je tedy inovativní svým přístupem ke kresbě diagramů.
U~složitých diagramů se však nutně ve zdrojovém kódu uživatel ztrácí, protože
nomnoml nenabízí žádné dělení čí kompozici tohoto kódu. S~vizuální reprezentací
grafu nelze přímo (např. myší) pracovat, veškeré rozložení a orientaci diagramu
tak musí uživatel nechat na algoritmu nástroje. Nástroj dále nenabízí ani
možnost pracovat s~několika diagramy najednou. Z~těchto důvodů určujeme produkt
jako vhodný pouze pro jednotlivce a tvorbu méně rozsáhlých UML diagramů.

\begin{figure}
  \centering
  \includegraphics[width=\textwidth]{../img/nomnoml.png}
  \caption{Tvorba UML diagramu v~nomnoml}
  \label{fig:nomnoml}
\end{figure}

\section{Závěr existujících řešení}

Přehled analýzy zmíněných existujících řešení je vidět v~tabulce
\ref{tab:existing-comparison}. % TODO doplnit

\newcommand{\tnote}[1]{\textsuperscript{#1}}
\begin{table}
  \begin{center}
    \begin{tabular}{r|cccc}
      \toprule
      název produktu           & \textbf{diagrams.net}                       & \textbf{drawSQL}                                & \textbf{ERDPlus} & \textbf{nomnoml}\\
      \midrule                 
      kategorie (vrstva)       & libovolné d.                          & logická                                         & konceptuální & omezené d. \\    
      serializovaný ex.     & ano                                         & ne\tnote{\ref{tab:ec:plan}}                     & ano & ano \\               
      rastrový ex.         & ano                                         & ano                                             & ano & ano  \\            
      vektorový ex.        & ano                                         & ne                                              & ne & ano \\              
      schématický ex. & ne                                          & SQL                                             & SQL & ne \\             
      zjednodušený ex.      & HTML, PDF                                   & ano\tnote{\ref{tab:scaffolding}} & ne & ne \\               
      poskytuje úložiště       & ne\tnote{\ref{tab:ec:external-storage}}     & ano                                             & ano & ano\tnote{\ref{tab:ec:urlsharing}} \\            
      paměť prohlížeče & ano                                         & ne                                               & ne & ano \\                
      \midrule[\heavyrulewidth]
      \end{tabular}
  \end{center}
  
  \footnotesize
  \begin{enumerate}[a.,ref=\alph*,noitemsep]
    \item plánovaná funkce \label{tab:ec:plan}
    \item využívá úložiště třetích stran \label{tab:ec:external-storage}
    \item platformně-specifický scaffolding -- automatické generování kódu pro specifické platformy a programovací jazyky \label{tab:scaffolding}
    \item diagram je uložen v~URL, pomocí které lze diagram sdílet \label{tab:ec:urlsharing}
  \end{enumerate}
  
  \caption{Srovnání existujících řešení}
  \label{tab:existing-comparison}
\end{table}