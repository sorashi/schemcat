\chapter*{Úvod}
\addcontentsline{toc}{chapter}{Úvod}

Vývoj informačních a databázových systémů se neustále posouvá vpřed.
Při jejich návrhu je zásadním krokem konceptuální modelování, které posléze vede na tvorbu databázového schématu.
Ke konceptuálnímu modelování existuje několik uznávaných přístupů, jako je \acrfull{er}~\cite{chen_er_1976} a \acrfull{uml}~\cite{omg_uml_2017}.
Tyto modely však byly navrhovány v dobách, kdy relační databáze vládly softwarovým systémům, poněvadž tabulky byly nejvhodnějším logickým modelem pro potřeby doby.

V posledních letech se v oboru začalo užívat rozličných databázových paradigmat.
Tím v praktické rovnině myslíme zejména různé modely na logické vrstvě modelování.
Každý takový model má svůj účel, své slabé a silné stránky.
Zkušený softwarový inženýr vybere pro každý systém tu nejvhodnější databázi s nejvhodnějším logickým modelem.
Někdy se dokonce modelů v jednom systému používá více najednou, aby každý mohl spravovat tu část dat systému, pro kterou je vhodný.

Mezi nejznámější logické modely databázových systémů patří
relační model (například databázový systém MySQL\footnote{\url{https://www.mysql.com/}}),
key-value model (Redis\footnote{\url{https://redis.io/}}),
wide column model (Apache Cassandra\footnote{\url{https://cassandra.apache.org/}}),
dokumentový model (MongoDB\footnote{\url{https://www.mongodb.com/}}) a 
grafový model (Neo4J\footnote{\url{https://neo4j.com/}}).
Kromě toho existují ještě např. fulltextové vyhledávací databáze (Elasticsearch\footnote{\url{https://www.elastic.co/}}) a
multi-modelové databáze (FaunaDB\footnote{\url{https://fauna.com/}}), které používají různé relační modely, respektive jich používají více najednou.

Naše stávající prostředky konceptuálního modelování nejsou mnohdy dostačující k modelování pro nerelační databáze.
Ukazuje se mimo jiné, že byly navrženy s relačními databázemi na mysli, a jsou tak bohužel svázány s modelem logické vrstvy.
Při konceptuálním modelování se chceme věnovat pouze vymezení a představě o části světa, na kterou vymezujeme svůj diskurz, nikoli na logickou strukturu dat v cílové databázi.

Utopií, které se snažíme dosáhnout, je mít jediný unifikovaný model, který má jediné rozhraní, jediný způsob modelování dat a jediný dotazovací jazyk v konceptuální vrstvě.
Jde o splynutí konceptuální a logické vrstvy do jedné, ve které se pracuje unifikovaným konceptuálním způsobem.
Až ve fyzické vrstvě se zvolí vhodná struktura podle potřeby dat.

Prvním hmatatelným krokem k přiblížení k tomuto cíli je schematická kategorie, kterou navrhli Martin Svoboda, Pavel Čontoš a Irena Holubová \cite{svoboda_categorical_2021}.
Jedná se o prostředek k modelování struktury dat, který je založený na teorii kategorií~\cite{eilenberg_generaltheory_1945}.

Cílem této práce je vytvořit softwarový nástroj, který umožní návrh konceptuálních schémat pomocí prostředků \acrshort{er} modelu a jeho převod na schematickou kategorii.
Přiblíží tak tento nový koncept softwarovým inženýrům a analytikům, kteří jsou zběhlí v modelování pomocí \acrshort{er}.
Nástroj dále usnadní výzkumníkům experimentování se schematickými kategoriemi.

Práce je rozdělena do několika částí.
Nejprve prozkoumáme a zanalyzujeme existující nástroje, které slouží ke konceptuálnímu modelování (Kapitola~\ref{chapter:existujici-nastroje}).
Dále definujeme a popíšeme upravené varianty \acrshort{er} a schematické kategorie navržené vedoucím práce, společně se způsobem vizualizace schematické kategorie a podrobnější motivací za schematickou kategorií (Kapitola~\ref{chapter:teorie}).
Posléze navrhneme specifikaci zamýšleného softwarového nástroje na tvorbu diagramů konceptuálních schémat pomocí \acrshort{er} s konverzí do schematické kategorie a vizualizací schematické kategorie (Kapitola~\ref{chapter:specifikace}).
Specifikace bude zahrnovat funkční a nefunkční požadavky, konceptuální model, procesy, koncept řešení, diagram tříd a případy užití.
Poté popíšeme implementaci a uvedeme dokumentaci tohoto softwarového nástroje včetně návodu k užití (Kapitola~\ref{chapter:implementace}).
Nakonec zhodnotíme výsledky práce.
