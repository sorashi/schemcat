%%% Tento soubor obsahuje definice různých užitečných maker a prostředí %%%
%%% Další makra připisujte sem, ať nepřekáží v ostatních souborech.     %%%

%%% Drobné úpravy stylu

% Tato makra přesvědčují mírně ošklivým trikem LaTeX, aby hlavičky kapitol
% sázel příčetněji a nevynechával nad nimi spoustu místa. Směle ignorujte.
\makeatletter
\def\@makechapterhead#1{
  {\parindent \z@ \raggedright \normalfont
      \Huge\bfseries \thechapter. #1
      \par\nobreak
      \vskip 20\p@
    }}
\def\@makeschapterhead#1{
  {\parindent \z@ \raggedright \normalfont
      \Huge\bfseries #1
      \par\nobreak
      \vskip 20\p@
    }}
\makeatother

% Custom -- úprava mezer řádků
\setlength{\parskip}{0.2em}
\setlist[enumerate]{partopsep=\parskip, topsep=\parskip, itemsep=0pt}
\setlist[itemize]{partopsep=\parskip, topsep=\parskip, itemsep=0pt}

% Toto makro definuje kapitolu, která není očíslovaná, ale je uvedena v obsahu.
\def\chapwithtoc#1{
  \chapter*{#1}
  \addcontentsline{toc}{chapter}{#1}
}

% Trochu volnější nastavení dělení slov, než je default.
\lefthyphenmin=2
\righthyphenmin=2

\iffinal
\else
  % Zapne černé "slimáky" na koncích řádků, které přetekly, abychom si
  % jich lépe všimli.
  \overfullrule=1mm
\fi

%%% Makra pro definice, věty, tvrzení, příklady, ... (vyžaduje baliček amsthm)

\theoremstyle{plain}
\newtheorem{theorem}{Věta}
\newtheorem{lemma}[theorem]{Lemma}
\newtheorem{statement}[theorem]{Tvrzení}
\newtheorem*{statement*}{Tvrzení}

\theoremstyle{definition}
\newtheorem{definition}{Definice}

\theoremstyle{remark}
\newtheorem*{corollary}{Důsledek}
\newtheorem*{note}{Poznámka}
\newtheorem*{example}{Příklad}

%%% Prostředí pro důkazy

\newenvironment{dukaz}{
  \par\medskip\noindent
  \textit{Důkaz}.
}{
  \newline
  \rightline{$\qedsymbol$}
}

%%% Prostředí pro sazbu kódu, případně vstupu/výstupu počítačových
%%% programů. (Vyžaduje balíček fancyvrb -- fancy verbatim.)

\DefineVerbatimEnvironment{code}{Verbatim}{fontsize=\small, frame=single}

%%% Prostor reálných, resp. přirozených čísel
\newcommand{\R}{\mathbb{R}}
\newcommand{\N}{\mathbb{N}}

%%% Užitečné operátory pro statistiku a pravděpodobnost
\DeclareMathOperator{\pr}{\textsf{P}}
\DeclareMathOperator{\E}{\textsf{E}\,}
\DeclareMathOperator{\var}{\textrm{var}}
\DeclareMathOperator{\sd}{\textrm{sd}}

%%% Příkaz pro transpozici vektoru/matice
\newcommand{\T}[1]{#1^\top}

%%% Vychytávky pro matematiku
\newcommand{\goto}{\rightarrow}
\newcommand{\gotop}{\stackrel{P}{\longrightarrow}}
\newcommand{\maon}[1]{o(n^{#1})}
\newcommand{\abs}[1]{\left|{#1}\right|}
\newcommand{\dint}{\int_0^\tau\!\!\int_0^\tau}
\newcommand{\isqr}[1]{\frac{1}{\sqrt{#1}}}

%%% Vychytávky pro tabulky
\newcommand{\pulrad}[1]{\raisebox{1.5ex}[0pt]{#1}}
\newcommand{\mc}[1]{\multicolumn{1}{c}{#1}}

%% Dočasná implementace pro zkratky
% \newacronym{gcd}{GCD}{Greatest Common Divisor}, \acrshort{gcd}
\newcommand{\newacronym}[3]{%
  \expandafter\newcommand\csname acrshort#1\endcsname{#2}%
  \expandafter\newcommand\csname acrlong#1\endcsname{#3}%
}
\newcommand{\acrshort}[1]{\csname acrshort#1\endcsname}
\newcommand{\acrlong}[1]{\csname acrlong#1\endcsname}
\newcommand{\acrfull}[1]{\csname acrlong#1\endcsname\ (\csname acrshort#1\endcsname)}

\newacronym{json}{JSON}{JavaScript Object Notation}
\newacronym{svg}{SVG}{Scalable Vector Graphics}
\newacronym{png}{PNG}{Portable Network Graphics}
\newacronym{gc}{GC}{garbage collection}
\newacronym{er}{ER}{Entity-Relationship}
\newacronym{uml}{UML}{Unified Modeling Language}
\newacronym{vsk}{VSK}{Vizualizace schematické kategorie}
\newacronym{url}{URL}{Uniform Resource Locator}
\newacronym{sql}{SQL}{Structured Query Language}
\newacronym{xml}{XML}{Extensible Markup Language}
\newacronym{http}{HTTP}{Hypertext Transfer Protocol}
\newacronym{mime}{MIME}{Multipurpose Internet Mail Extensions}
\newacronym{dom}{DOM}{Document Object Model}

% často použité věci v textu
\newcommand{\OO}{\mathcal O}
\newcommand{\MM}{\mathcal M}
\newcommand{\sig}{\mathit{sig}}
\newcommand{\dom}{\mathit{dom}}
\newcommand{\cod}{\mathit{cod}}
\newcommand{\minn}{\mathit{min}}
\newcommand{\maxx}{\mathit{max}}
\newcommand{\dup}{\mathit{dup}}
\newcommand{\ord}{\mathit{ord}}
\newcommand{\dir}{\mathit{dir}}
\newcommand{\name}{\mathit{name}}

\newcommand{\xmark}{\XSolidBrush}
\newcommand{\cmark}{\CheckmarkBold}
\newcommand{\set}[1]{\left\{#1\right\}}
\newcommand{\zero}{\texttt{0}}
\newcommand{\one}{\texttt{1}}
\newcommand{\many}{\texttt{*}}
\newcommand{\zeroone}{(\zero{}, \one{})}
\newcommand{\oneone}{(\one{}, \one{})}
\newcommand{\zeromany}{(\zero{}, \many{})}
\newcommand{\onemany}{(\one{}, \many{})}
\newcommand{\true}{\texttt{true}}
\newcommand{\false}{\texttt{false}}
%%%% end convenience commands
