\chapter{Implementace}

V této kapitole popíšeme postup implementace aplikace na kresbu diagramů \acrfull{er}, \acrfull{vsk} a syrových schematických kategorií.

\section{Použité technologie}

K vývoji jsme použili framework React~\cite{react_2023}.
React zjednodušuje tvorbu webových uživatelských rozhraní a celých webových aplikací tím, že umožňuje vytvářet celé komponenty a skládat z nich další.

Programovací jazyk jsme zvolili TypeScript~\cite{microsoft_typescriptjavascript_2023}, který k JavaScriptu přidává statické typování.
Výsledkem kompilace je JavaScript, který lze interpretovat prohlížečem.

Pro správu globálního stavu aplikace jsme použili knihovnu Zustand~\cite{daishikato_zustand_2023}.
V React totiž každá komponenta má svůj stav, který předává jako vlastnosti svým dětem.
Úprava stavu musí být uskutečněna dětmi pomocí události, kterou odebírá rodič.
V našem případě, kdy v jednom diagramu bude mnoho volných komponent (konstruktů), které mezi sebou potřebují komunikovat tento přístup není vhodný.
Přestože v React existují způsoby pro sdílení globálního stavu (React Context), jejich metoda sdílení stavu pro naši aplikaci není vhodná.
Zustand umožňuje vytvořit React Hook, pomocí kterého můžeme v libovolné komponentě číst i upravovat jeden globální stav.

Knihovna Zundo~\cite{kornoelje_zundo_2023} umožňuje globální stav vracet o stav zpět a dopředu (undo/redo).
Ovšem zachytává i nepatrné změny a tak se sledování stavů musí vhodně vypínat a zapínat.

Dále využijeme knihovnu Immer~\cite{michelweststrate_immer_2023}, která umožňuje v React stav mutovat.
Při použití Reactu by se totiž stav přímo mutovat neměl, protože by nebyla zaregistrována jeho změna, React by na ni nemohl náležitě zareagovat a došlo by k nekonzistenci.
React změnu stavu detekuje referenčním (ne hlubokým) porovnáním předchozího a nového stavu komponenty.
Při každé změně stavu se proto běžně musí vytvořit nový objekt.
Pokud je objekt hluboce zanořený, je tvorba nového objektu velmi explicitní a zdlouhavá na programování.
React by se správně měl používat bez zanořených stavů, takže každá komponenta se stará o svůj minimální stav.
Protože však používáme jeden složitý globální stav, Immer je vhodná pomoc.
Immer udělá většinu práce za nás tím, že vytvoří tzv. draft (pracovní verze) stavu a sleduje změny, které na něm program dělá.
Pomocí těchto změn pak vytvoří novou instanci stavu.
Porovnání práce se stavem bez a s knihovnou Immer lze vidět v Kódu~\ref{code:immer}.

Další důležitou použitou knihovnou je class-transformer~\cite{attilaolah_classtransformer_2023}.
V JavaScriptu, resp. v TypeScriptu existují dva druhy objektů -- \emph{plain} a \emph{class} objekt.
Při serializaci class objektů do JSON~\cite{tc39group_jsondata_2017} a zpět ztrácí objekt své metody, metody předků a další informace o třídě, z které byla zkonstruována jeho instance.
Knihovna class-transformer umožňuje mezi těmito typy objektů převádět a my tak můžeme používat instanční metody na objektech deserializovaných z JSON.
Knihovna nefunguje perfektně pro všechny druhy datových složek, a pokládá na třídy další určité restrikce.

Pro zjednodušení práce s CSS styly používáme knihovnu Tailwind CSS~\cite{tailwindlabs_tailwindcss_2020}.
Tailwind CSS proráží metodu nastavování stylů pomocí tzv. utilitních tříd.
Velmi dobře se kombinuje s frameworkem React, protože místo psaní stylů do CSS souborů s arbitrárními názvy, Tailwind CSS umožňuje styl komponent upravovat přímo v jejich atributu \texttt{class}.
Například, pokud chceme nastavit okraje ovládacího prvku na 24 pixelů, použijeme třídu \texttt{m-6}, kde \texttt{m} je zkratka ze slova margin (okraj) a \texttt{6} znamená \enquote{šestá úroveň}.
Úrovně různých nastavení mají předdefinované (ale nastavitelné) hodnoty a umožňují konzistentní styl -- délky, barvy i celé kombinace několika vlastností.
Při kompilaci aplikace se ve výsledném balíčku CSS zakomponují z Tailwind CSS pouze třídy, které byly v aplikaci opravdu použity.

Pro kompilaci a vytvoření výsledného balíčku statické webové aplikace používáme nástroj Vite (čte se francouzsky, [vít]).
Výstupem této kompilace je ideálně pouze jediný HTML soubor, jediný JavaScript soubor a jediný CSS soubor, které jsou připraveny na nasazení.
Vite totiž přeloží TypeScript do JavaScriptu, najde a propojí všechny importované moduly a spojí vše do jediného souboru.
\mcomment{Stačí pro použité technologie takhle strana? Mám ještě o čem psát, konkrétně třeba knihovny na úpravu datových chunků PNG souborů.}

Stěžejní knihovnou, která poskytuje prvky uživatelského rozhraní, je FlexLayout~\cite{_flexlayout_2023}.
Jedná se o rozhraní s okny, která se dají přesouvat, roztahovat a přepínat záložkami.
Tato knihovna byla zvolena, protože je flexibilní a uživatel si své pracovní prostředí díky jejím prvkům může libovolně upravit.

\begin{lstlisting}[language=JavaScript,float=htb,caption=Použití knihovny Immer,label=code:immer]
// tvorba stavu v React
const [state, setState] = useState({
  person: {
    name: "Name",
    age: 30,
    inventory: {
      description: "Description",
      productIds: [1,2,3]
    }
  }
});

// aktualizace stavu bez knihovny Immer
setState(prevState => ({
  ...prevState,
  person: {
    ...prevState.person,
    inventory: {
      ...prevState.person.inventory,
      productIds: [...prevState.person.inventory.productIds, 4]
    }
  }
}))

// aktualizace stavu s knihovnou Immer
setState(produce(draft => {
  draft.person.inventory.productIds.push(4)
}))
\end{lstlisting}
