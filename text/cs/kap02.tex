\chapter{Specifikace}

Sommerville~\cite{sommerville_software_2011} % strana 6
uvádí fundamentální aktivity softwarového inženýrství: specifikace, vývoj, validace a evoluce.
Dále definuje softwarovou specifikaci jako aktivitu, při které zákazníci a inženýři definují software, který má být vyprodukován, a omezení na jeho provoz. % strana 9

V~této kapitole definujeme software, který budeme tvořit, pomocí požadavků uživatele, případů užití, diagramu tříd v~konceptuální vrstvě a procesů.

\section{Požadavky}

Požadavky na softwarový systém jsou popisy toho, co by měl systém dělat -- služby, které poskytuje, a omezení na jeho provoz.
Tyto požadavky by měly reflektovat potřeby zákazníka na systém a jeho účel, uvádí Sommerville~\cite[s.~83]{sommerville_software_2011}. % strana 83

Požadavky na funkce systému nazýváme funkční.
Požadavky na provoz systému a jeho omezení nazýváme nefunkční.

Dále uvedeme vybrané funkční a nefunkční požadavky z~pohledu uživatele systému.

\subsection{Funkční požadavky}
\newlist{enumfp}{enumerate}{1}
\setlist[enumfp]{label=\textbf{FP-\arabic*},ref=FP-\arabic*}

\subsubsection*{Projekt}

\begin{enumfp}
    \item V~systému bude možné vytvořit nový projekt.
    \item Projekt bude možné uložit.
    \item Projekt bude možné načíst.
    \item Projekt bude možné pojmenovat pro odlišení od ostatních projektů.
    \item Jednotlivé diagramy bude možné exportovat do rastrového i vektorového formátu.
    \item Do těchto exportovaných formátů bude volitelně možné vložit projekt, který z~nich pak bude možné načíst.
        Tímto bude projekt možné otevřít jak v~prohlížeči obrázků (a zobrazit rastrově nebo vektorově diagram), tak v~našem systému a pokračovat v~práci.
    \item Zobrazení každého diagramu bude možné posouvat myší.
    \item Zobrazení každého diagramu bude možné přibližovat a oddalovat kolečkem myši.
\end{enumfp}

\subsubsection*{ER Diagram}

\begin{enumfp}[resume]
    \item Do diagramu bude možné přidat entitní typ.
    \item Do diagramu bude možné přidat vztahový typ.
    \item Do diagramu bude možné přidat atribut.
    \item Mezi entitním a vztahovým typem bude možné vytvořit spojení.
    \item Mezi entitním typem a atributem bude možné vytvořit spojení.
    \item Bude možné vytvořit ISA hierarchii mezi entitními typy. 
    \item Vlastnosti všech objektů bude možné měnit (popisky, typ, pozice).
    \item Objekty bude možné posunovat držením levého tlačítka myši a tažením.
    \item Všechny objekty bude možné zvolit levým tlačítkem myší.
    \item Při držení klávesy \keys{\ctrl} bude možné zvolit více objektů najednou postupným klikáním levého tlačítka myši.
    \item Veškeré objekty bude možné z~diagramu mazat alespoň klávesou \keys{Delete}.
    \item Uživatel bude moct využít předpřiprvené konstrukty, které bude možné vložit do diagramu.
        Například se může jednat o~ISA hierarchie s~předpřipravenými entitami.
\end{enumfp}

\subsection{Nefunkční požadavky}
\newlist{enumnfp}{enumerate}{1}
\setlist[enumnfp]{label=\textbf{NFP-\arabic*},ref=NFP-\arabic*}

\begin{enumnfp}
    \item Aplikaci bude možné používat na všech běžných desktopových operačních systémech.
    \item Nesmí dojít ke ztrátě práce při náhlém ukončení aplikace kvůli interním či externím vlivům.
    \item Aplikaci bude možné používat i při výpadku internetového připojení.
\end{enumnfp}
