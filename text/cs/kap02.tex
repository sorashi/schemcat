\chapter{Specifikace}

Sommerville~\cite{sommerville_software_2011} % strana 6
uvádí fundamentální aktivity softwarového inženýrství: specifikace, vývoj, validace a evoluce.
Dále definuje softwarovou specifikaci jako aktivitu, při které zákazníci a inženýři definují software, který má být vyprodukován, a omezení na jeho provoz. % strana 9

V této kapitole definujeme software, který budeme tvořit, pomocí požadavků uživatele, případů užití, diagramu tříd v konceptuální vrstvě a procesů.

\section{Požadavky}

Požadavky na softwarový systém jsou popisy toho, co by měl systém dělat -- služby, které poskytuje, a omezení na jeho provoz.
Tyto požadavky by měly reflektovat potřeby zákazníka na systém a jeho účel, uvádí Sommerville~\cite[s.~83]{sommerville_software_2011}. % strana 83

Požadavky na funkce systému nazýváme funkční.
Požadavky na provoz systému a jeho omezení nazýváme nefunkční.

Dále uvedeme vybrané funkční a nefunkční požadavky z pohledu uživatele systému.

\subsection{Funkční požadavky}
\newlist{enumfp}{enumerate}{1}
\setlist[enumfp]{label=\textbf{FP-\arabic*},ref=FP-\arabic*}

\subsubsection*{Projekt}

\begin{enumfp}
    \item V systému bude možné vytvořit nový projekt.
    \item Projekt bude možné uložit.
    \item Projekt bude možné načíst.
    \item Jednotlivé diagramy bude možné exportovat do rastrového i vektorového obrázku.
\end{enumfp}

\subsubsection*{ER Diagram}

\begin{enumfp}[resume]
    \item 
\end{enumfp}

\subsection{Nefunkční požadavky}
\newlist{enumnfp}{enumerate}{1}
\setlist[enumnfp]{label=\textbf{NFP-\arabic*},ref=NFP-\arabic*}

\begin{enumnfp}
    \item Aplikaci bude možné používat na všech běžných desktopových operačních systémech.
    \item Nesmí dojít ke ztrátě práce při pádu aplikace kvůli interním či externím vlivům.
    \item Aplikaci bude možné používat bez internetového připojení.
\end{enumnfp}
