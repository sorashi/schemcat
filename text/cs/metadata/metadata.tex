% Název práce v jazyce práce (přesně podle zadání)
\def\NazevPrace{\input{metadata/title-cz.txt}}

% Název práce v angličtině
\def\NazevPraceEN{\input{metadata/title-en.txt}}

% Jméno autora
\def\AutorPrace{Dennis Pražák}

% Rok odevzdání
\def\RokOdevzdani{2023}

% Název katedry nebo ústavu, kde byla práce oficiálně zadána
% (dle Organizační struktury MFF UK, případně plný název pracoviště mimo MFF)
\def\Katedra{Katedra softwarového inženýrství}
\def\KatedraEN{Department of Software Engineering}

% Jedná se o katedru (department) nebo o ústav (institute)?
\def\TypPracoviste{Katedra}
\def\TypPracovisteEN{Department}

% Vedoucí práce: Jméno a příjmení s~tituly
\def\Vedouci{RNDr.~Martin Svoboda, Ph.D.}

% Pracoviště vedoucího (opět dle Organizační struktury MFF)
\def\KatedraVedouciho{\Katedra}
\def\KatedraVedoucihoEN{\KatedraEN}

% Studijní program a obor
\def\StudijniProgram{Informatika }
\def\StudijniObor{IPP2}

% Nepovinné poděkování (vedoucímu práce, konzultantovi, tomu, kdo
% zapůjčil software, literaturu apod.)
\def\Podekovani{%
  Děkuji vedoucímu RNDr.~Martinu Svobodovi,~PhD., za odborné vedení práce a všechny poskytnuté rady a podněty.
}

% Abstrakt (doporučený rozsah cca 80-200 slov; nejedná se o zadání práce)
\def\Abstrakt{\input{metadata/abstrakt-cz.txt}}
\def\AbstraktEN{%%% Šablona pro jednoduchý soubor formátu PDF/A, jako treba samostatný abstrakt práce.

\documentclass[12pt]{report}

\usepackage[a4paper, hmargin=1in, vmargin=1in]{geometry}
\usepackage[a-2u]{pdfx}
\usepackage{polyglossia}
\setmainlanguage{czech}
\setotherlanguages{english}
\usepackage{lmodern}
\usepackage{textcomp}
\usepackage{microtype}
\usepackage[mono=true]{libertinus-otf}

\begin{document}

%%% Šablona pro jednoduchý soubor formátu PDF/A, jako treba samostatný abstrakt práce.

\documentclass[12pt]{report}

\usepackage[a4paper, hmargin=1in, vmargin=1in]{geometry}
\usepackage[a-2u]{pdfx}
\usepackage{polyglossia}
\setmainlanguage{czech}
\setotherlanguages{english}
\usepackage{lmodern}
\usepackage{textcomp}
\usepackage{microtype}
\usepackage[mono=true]{libertinus-otf}

\begin{document}

%%% Šablona pro jednoduchý soubor formátu PDF/A, jako treba samostatný abstrakt práce.

\documentclass[12pt]{report}

\usepackage[a4paper, hmargin=1in, vmargin=1in]{geometry}
\usepackage[a-2u]{pdfx}
\usepackage{polyglossia}
\setmainlanguage{czech}
\setotherlanguages{english}
\usepackage{lmodern}
\usepackage{textcomp}
\usepackage{microtype}
\usepackage[mono=true]{libertinus-otf}

\begin{document}

\input{metadata/abstrakt-en.txt}

\end{document}


\end{document}


\end{document}
}

% 3 až 5 klíčových slov (doporučeno), každé uzavřeno ve složených závorkách
\def\KlicovaSlova{\input{metadata/keywords-cz.txt}}
\def\KlicovaSlovaEN{\input{metadata/keywords-en.txt}}
